\hrule
\vspace{1mm}
\addcontentsline{toc}{subsection}{grep}
\headerline{\large grep%
    \index{grep}%
    \index{search!in file|see{grep}}%
    \index{file!search|see{grep}}}%
    
    {\emph{file pattern searcher}}{}
\hrule
\vspace{4mm}

\paragraph{Description}
\indentpar \raggedright \textrm{Searches within the file(s) given for a specified string. More advanced searches can be done using regular expressions.}\\

\paragraph{Usage}
\indentpar grep \textit{[options] pattern file [file2...]}

\paragraph{Arguments}
\indentpar \argumentline{pattern}{the \texttt{pattern} to look for in the given file(s)}\\
\indentpar \argumentline{file}{the file to be searched for instances of \texttt{pattern}}

\paragraph{Output}
\indentpar \textrm{Prints out each line containing a match from within \texttt{file}.}

\paragraph{Useful options}
\indentpar \optionline{-c}{outputs the number of lines containing \texttt{string}}\\
\indentpar \optionline{-C}{print two lines before and after the match}
\indentpar \optionline{-e \textit{pattern}}{enables the use of regular expressions}\\
\indentpar \optionline{-f \textit{file}}{a file containing a list of \texttt{pattern}; one per line}\\
\indentpar \optionline{-F \textit{"string"}}{specifies \texttt{pattern} should be interpreted literally}\\
\indentpar \optionline{-i}{ignore case}\\
\indentpar \optionline{-l}{list input files that do match}\\
\indentpar \optionline{-L}{list input files that do not match}\\
\indentpar \optionline{-m \textit{\#}}{stop searching after finding \texttt{\#} matches}\\
\indentpar \optionline{-n}{print the line number for each match}\\
\indentpar \optionline{-r}{recursively search through directories}
\indentpar \optionline{-v}{print non-matching lines}\\
\indentpar \optionline{-w}{matches must be full-word matches}


\paragraph{Examples}

\indentpar grep pattern file.txt\\
\indentpar grep -Fwnc "pattern" file.txt file2.txt\\
\indentpar grep -r pattern directory\\
%\indentpar grep -e 

\vspace{20mm}
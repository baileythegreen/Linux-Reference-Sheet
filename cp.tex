\hrule
\vspace{1mm}
\addcontentsline{toc}{subsection}{cp}
\headerline{\large cp\index{cp}\index{copy|see{cp}}}{\emph{copy a file}}{}
\hrule
\vspace{4mm}

\paragraph{Description}
\indentpar \raggedright \textrm{Makes a copy of one or more specified files or directories. There are several ways this can be used, determined by the types of arguments given: copy a file locally, with a different name; copy one or more files to a new location, retaining the original names; copy a directory and its contents locally, with a different name; copy one or more directories and their contents to a new location, retaining the original names.}\\

\paragraph{Usage}
\indentpar cp \textit{file newfile}\\
\indentpar cp \textit{file [file2...] dest\_directory}\\
\indentpar cp \textit{directory newdirectory}\\
\indentpar cp \textit{directory [directory2...] dest\_directory}

\paragraph{Arguments}
\indentpar \argumentline{file}{the file to be copied}\\
\indentpar \argumentline{newfile}{the name of the copy being created}\\
\indentpar \argumentline{file2...}{additional files to be copied (optional)}\\
\indentpar \argumentline{dest\_directory}{the destination location for the copies}
\indentpar \argumentline{directory}{the directory to be copied}\\
\indentpar \argumentline{newdirectory}{the name of the copy being created}\\
\indentpar \argumentline{directory2...}{additional directories to be copied (optional)}

\paragraph{Useful options}
\indentpar \optionline{-n}{do not overwrite an existing file}\\


\paragraph{Examples}

\indentpar cp -n file.txt newfile.txt\\
\indentpar cp directory/ another/directory/\\
\indentpar cp * path/to/directory/\\

\vspace{20mm}
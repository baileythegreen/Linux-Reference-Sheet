\hrule
\vspace{1mm}
\addcontentsline{toc}{subsection}{head}
\headerline{\large head\index{head}}{\emph{print the beginning of the file}}{}
\hrule
\vspace{4mm}

\paragraph{Description}
\indentpar \raggedright \textrm{By default, prints out the first ten lines of each file given.}\\

\paragraph{Usage}
\indentpar head [options] \textit{file [file2...]}

\paragraph{Arguments}
\indentpar \argumentline{file}{the file to print the top of (may be piped)}\\
\indentpar \argumentline{file2...}{additional files (optional)}

\paragraph{Output}
\indentpar \textrm{Prints the first ten lines of the file to standard output.}

\paragraph{Useful options}
\indentpar \optionline{-n \textit{\#}}{specifies the number of lines to print}


\paragraph{Examples}

\indentpar head file.txt\\
\indentpar head -n 20 file.txt file2.txt

\vspace{20mm}
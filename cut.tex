\hrule
\vspace{1mm}
\addcontentsline{toc}{subsection}{cut}
\headerline{\large cut\index{cut}}{\emph{cut out vertical sections of a file}}{}
\hrule
\vspace{4mm}

\paragraph{Description}
\indentpar \raggedright \textrm{Extracts the specified characters or fields from each line of a file, in turn, returning a series of columns.}\\

\paragraph{Usage}
\indentpar cut \textit{[options] file}

\paragraph{Arguments}
\indentpar \argumentline{file}{the input file (may be piped)}

\paragraph{Output}
\indentpar \textrm{The specified portions of the file are printed to standard output as columns.}

\paragraph{Useful options}
\indentpar \optionline{-c list}{\texttt{list} specifies the desired character positions}\\
\indentpar \optionline{-d string}{\texttt{string} represents the field delimiter; tab by default}\\
\indentpar \optionline{-f list}{\texttt{list} specifies the desired field positions}\\
\indentpar \optionline{-s}{suppress lines with no field delimiters}
\indentpar \optionline{--complement}{returns the inverse of the request}


\paragraph{Examples}

\indentpar cut -d , -f 2,3,10 file.csv\\
\indentpar cut -c 5-10,18-27 --complement file.tsv

\vspace{20mm}
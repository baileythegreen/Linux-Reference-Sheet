\hrule
\vspace{1mm}
\addcontentsline{toc}{subsection}{whatis}
\headerline{\large whatis%
    \index{whatis}%
    \index{help|see{man, whatis}}
}{\emph{briefly describe what a command does}}{}
\hrule
\vspace{4mm}

\paragraph{Description}
\indentpar \raggedright \textrm{Searches within names and one-line descriptions of commands' functions for the searchterm specified. Useful to learn what a command does, or see other commands with similar functions.}\\

\paragraph{Usage}
\indentpar whatis \textit{searchterm}

\paragraph{Arguments}
\indentpar \argumentline{searchterm}{a known command name or a string}

\paragraph{Output}
\indentpar \textrm{A list of command names and their one-line descriptions. A name/description pair will be printed if either contains \texttt{searchterm} as a full-word match. This is different to \texttt{apropos}, where \texttt{searchterm} may occur anywhere in the documentation, and may also be a substring in a larger word.}


\paragraph{Examples}

\indentpar whatis mkdir\\
\indentpar whatis tail

\vspace{20mm}
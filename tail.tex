\hrule
\vspace{1mm}
\addcontentsline{toc}{subsection}{tail}
\headerline{\large tail\index{tail}}{\emph{print the end of the file}}{}
\hrule
\vspace{4mm}

\paragraph{Description}
\indentpar \raggedright \textrm{By default, prints out the last ten lines of each file given.}\\

\paragraph{Usage}
\indentpar tail \textit{[options] file [file2...]}

\paragraph{Arguments}
\indentpar \argumentline{file}{the file to print the bottom of (may be piped)}\\
\indentpar \argumentline{file2...}{additional files (optional)}

\paragraph{Output}
\indentpar \textrm{Prints the last then lines of the file to standard output.}

\paragraph{Useful options}
\indentpar \optionline{-f}{continually update the end of the file}
\indentpar \optionline{-n \textit{\#}}{print the last \# lines}\\
\indentpar \optionline{-n +\#}{print the file from line \# to the end}\\
\indentpar \optionline{-q}{ omit headers if there are multiple files}\\
\indentpar \optionline{-r}{print the lines in reverse}\\



\paragraph{Examples}

\indentpar tail file.txt\\
\indentpar tail -qn +50 file.txt file2.txt

\vspace{20mm}
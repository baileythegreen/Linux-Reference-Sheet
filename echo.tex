\hrule
\vspace{1mm}
\addcontentsline{toc}{subsection}{echo}
\headerline{\large echo\index{echo}\index{evaluate|see{echo}}}{\emph{write arguments to standard output}}{}
\hrule
\vspace{4mm}

\paragraph{Description}
\indentpar \raggedright \textrm{Writes the value of its arguments to the standard output. If variables or expressions are included, they are first evaluated, then written out.}\\

\paragraph{Usage}
\indentpar echo \textit{[options] argument [argument2...]}\\

\paragraph{Arguments}
\indentpar \argumentline{argument}{argument to evaluate and write to the shell}\\
\indentpar \argumentline{argument2...}{additional arguments to be written out}

\paragraph{Output}
\indentpar \textrm{Prints the evaluated argument(s) to the shell in the order in which they are given. By default, the output is followed by a newline character.}

\paragraph{Useful options}
\indentpar \optionline{-e}{enables evaluation of escaped characters}\\
\indentpar \optionline{-n}{omits the newline character at the end of the output}


\paragraph{Examples}

\indentpar echo "this will be printed"\\
\indentpar x=5\\
\indentpar echo \$x\\
\indentpar echo "The date is: \$(date +\%D)"

\vspace{20mm}
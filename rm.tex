\hrule
\vspace{1mm}
\addcontentsline{toc}{subsection}{rm}
\headerline{\large rm\index{rm}}{\emph{remove files}}{}
\hrule
\vspace{4mm}

\paragraph{Description}
\indentpar \raggedright \textrm{Remove the specified files. May also be used to remove directories using the \texttt{-r} option; this does not provide a warning in the case of non-empty directories.}\\

\paragraph{Usage}
\indentpar rm \textit{file [file2...]}

\paragraph{Arguments}
\indentpar \argumentline{file}{the file to remove}\\
\indentpar \argumentline{file2...}{additional files to remove (optional)}

\paragraph{Useful options}
\indentpar \optionline{-i}{request confirmation before removing each file}\\
\indentpar \optionline{-r}{recursively remove a directory and its contents}

\paragraph{Examples}

\indentpar rm file.txt\\
\indentpar rm -r directory

\vspace{20mm}
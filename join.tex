\hrule
\vspace{1mm}
\addcontentsline{toc}{subsection}{join}
\headerline{\large join\index{join}}{\emph{merge two files}}{}
\hrule
\vspace{4mm}

\paragraph{Description}
\indentpar \raggedright \textrm{Combine two files based on a common field. Similar to commands like \texttt{merge()} in R and \texttt{Pandas.DataFrame.merge()} in python. Performs an 'inner' merge, only outputting lines for key values appearing in both input files.  A specific field may be set as the key, and specific columns for the output can be set using the -o option and a list of fields given in file\#.field\# notation.}\\

\paragraph{Usage}
\indentpar join \textit{[options] file1 file2}

\paragraph{Arguments}
\indentpar \argumentline{file1}{one of the files to be joined}\\
\indentpar \argumentline{file2}{the other file to be joined}

\paragraph{Output}
A line containing the requested columns from each file for each matching key value.

\paragraph{Useful options}
\indentpar \optionline{-1 \textit{field\#}}{use \texttt{field\#} from \texttt{file1} as the key}\\
\indentpar \optionline{-2 \textit{field\#}}{use \texttt{field\#} from \texttt{file2} as the key}\\
\indentpar \optionline{-e \textit{string}}{replace empty output fields with \texttt{string}}\\
\indentpar \optionline{-o \textit{list}}{the fields that will be output from each file}\\
\indentpar \optionline{-t \textit{char}}{use \texttt{char} as the field delimiter; default is tab or space}\\
\indentpar \optionline{-v \textit{\#}}{only print the unpaired lines from file\#}


\paragraph{Examples}

\indentpar join file1.tsv file2.tsv\\
\indentpar join -2 5 -o 1.1,1.4,2.1,2.2,2.5 file1.tsv file2.tsv

\vspace{20mm}
\hrule
\vspace{1mm}
\addcontentsline{toc}{subsection}{column}
\headerline{\large column\index{column}}{\emph{columnate input}}{}
\hrule
\vspace{4mm}

\paragraph{Description}
\indentpar \raggedright \textrm{Prints input in columnar format. If input is already organised into columns, as in a table, these columns are maintained. In cases of columns with variable content length, this produces clearly-defined columns.}\\

\paragraph{Usage}
\indentpar column \textit{[options] file}

\paragraph{Arguments}
\indentpar \argumentline{file}{input to print as columns (may be piped)}

\paragraph{Output}
\indentpar \textrm{Prints the input as a series of tab-delimited columns.}

\paragraph{Useful options}
\indentpar \optionline{-c \textit{c}}{set width of output to \texttt{c} columns}\\
\indentpar \optionline{-s \textit{string}}{separate columns with \texttt{string}}\\
\indentpar \optionline{-t}{format columns like a table; useful for tabular data}



\paragraph{Examples}

\indentpar column -s "\t|\t" file.tsv\\
\indentpar head file.tsv | column -t

\vspace{20mm}
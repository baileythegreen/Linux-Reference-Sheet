\hrule
\vspace{1mm}
\addcontentsline{toc}{subsection}{mv}
\headerline{\large mv
    \index{mv}%
    \index{rename|see{mv}}
    \index{change!\textrm{filename}|see{mv}}
}{\emph{move or rename files}}{}
\hrule
\vspace{4mm}

\paragraph{Description}
\indentpar \raggedright \textrm{Moves one or more specified files or directories. There are several ways this can be used, determined by the types of arguments given: rename a file (move locally to a new name); move a file to a new directory, retaining the original name; rename a directory (move locally to a new name); and move a directory and its contents to a new location, retaining the original names.}\\

\paragraph{Usage}
\indentpar mv \textit{file renamed\_file}\\
\indentpar mv \textit{file dest\_directory}\\
\indentpar mv \textit{directory renamed\_dir}\\
\indentpar mv \textit{directory dest\_directory}

\paragraph{Arguments}
\indentpar \argumentline{file}{the file to be moved/renamed}\\
\indentpar \argumentline{renamed\_file}{the new name for the file}\\
\indentpar \argumentline{dest\_directory}{the new location for the file/directory}
\indentpar \argumentline{directory}{directory to be moved/renamed}\\
\indentpar \argumentline{renamed\_dir}{the new name for the directory}


\paragraph{Useful options}
\indentpar \optionline{-i}{warn if the \texttt{mv} will overwrite an existing file}
\indentpar \optionline{-n}{do not overwrite an existing file}

\paragraph{Examples}
\indentpar mv oldfile.txt newfile.txt\\
\indentpar mv -n file.txt path/to/dest\_directory\\
\indentpar mv -i directory path/to/dest\_directory

\vspace{20mm}